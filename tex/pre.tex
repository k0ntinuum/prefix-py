

\documentclass{article}
\usepackage[utf8]{inputenc}
\usepackage{setspace}
\usepackage{ mathrsfs }
\usepackage{amssymb} %maths
\usepackage{amsmath} %maths
\usepackage[margin=0.2in]{geometry}
\usepackage{graphicx}
\usepackage{ulem}
\setlength{\parindent}{0pt}
\setlength{\parskip}{10pt}
\usepackage{hyperref}
\usepackage[autostyle]{csquotes}

\usepackage{cancel}
\renewcommand{\i}{\textit}
\renewcommand{\b}{\textbf}
\newcommand{\q}{\enquote}
%\vskip1.0in



\begin{document}

\begin{huge}



{\setstretch{0.0}{

\b{Pre} is a symmetric cryptosystem based primarily on prefix codes, also known as  instantaneous codes. This makes the encrypted output difficult to tokenize. It is often the case that a single plaintext symbol becomes several ciphertext symbols. So the output tends to be longer than the input.

Yet this crypto system also occasionally compresses its input, replacing several symbols with just one. While this feature could be removed, leaving the system intact, it's a nice complement that only increases the unpredictable relationship between the length of the plain and the ciphertext. 

[I will elaborate on this soon.]



}}
\end{huge}
\end{document}
